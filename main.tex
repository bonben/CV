%%%%%%%%%%%%%%%%%%%%%%%%%%%%%%%%%%%%%%%%%
% "ModernCV" CV and Cover Letter
% LaTeX Template
% Version 1.1 (9/12/12)
%
% This template has been downloaded from:
% http://www.LaTeXTemplates.com
%
% Original author:
% Xavier Danaux (xdanaux@gmail.com)
%
% License:
% CC BY-NC-SA 3.0 (http://creativecommons.org/licenses/by-nc-sa/3.0/)
%
% Important note:
% This template requires the moderncv.cls and .sty files to be in the same 
% directory as this .tex file. These files provide the resume style and themes 
% used for structuring the document.
%
%%%%%%%%%%%%%%%%%%%%%%%%%%%%%%%%%%%%%%%%%

%----------------------------------------------------------------------------------------
%	PACKAGES AND OTHER DOCUMENT CONFIGURATIONS
%----------------------------------------------------------------------------------------

\documentclass[11pt,a4paper,sans]{moderncv} % Font sizes: 10, 11, or 12; paper sizes: a4paper, letterpaper, a5paper, legalpaper, executivepaper or landscape; font families: sans or roman
\usepackage{standalone}
\moderncvstyle{classic} % CV theme - options include: 'casual' (default), 'classic', 'oldstyle' and 'banking'
\moderncvcolor{blue} % CV color - options include: 'blue' (default), 'orange', 'green', 'red', 'purple', 'grey' and 'black'

\usepackage{lipsum} % Used for inserting dummy 'Lorem ipsum' text into the template

\usepackage[scale=0.85]{geometry} % Reduce document margins
%\setlength{\hintscolumnwidth}{3cm} % Uncomment to change the width of the dates column
%\setlength{\makecvtitlenamewidth}{10cm} % For the 'classic' style, uncomment to adjust the width of the space allocated to your name

%\usepackage[utf8]{inputenc}

%\usepackage{booktabs}
\usepackage{fontawesome}
\usepackage{marvosym} % For cool symbols.
%\usepackage{hyperref}



%----------------------------------------------------------------------------------------
%	NAME AND CONTACT INFORMATION SECTION
%----------------------------------------------------------------------------------------

\firstname{MATHIEU} % Your first name
\familyname{LEONARDON} % Your last name

% All information in this block is optional, comment out any lines you don't need
\title{Associate Professor\\ \\ Algorithm-Architecture Adequation \\ Deep Learning Compression}
\address{MEE Department}{IMT Atlantique\\ born March 18th, 1987, French\\}
\mobile{(+33) 229001384}


%\fax{(000) 111 1113}
 
%\social{github}{stefano-bragaglia}
\email{mathieu.leonardon@imt-atlantique.fr} 



\homepage{www.mathieuleonardon.com/}{My Webpage}

% social link \faGithub, \faSkype, \faLinkedin,\faStackExchange, and \faStackOverflow
\extrainfo{
    \faGithub\href{https://github.com/bonben}{ Github} \quad
    \faLinkedin\href{https://bit.ly/3us0P9P}{ Linkedin} \quad
    % \faSkype\href{https://skype.com/abc}{Skype}
    }



%\social[linkedin][www.linkedin.com]{name}
% The first argument is %the url for the clickable link, the second argument is the url displayed in the %template - this allows special characters to be displayed such as the tilde in this %example

\photo[70pt][0.3pt]{picture} % The first bracket is the picture height, the second is %the thickness of the frame around the picture (0pt for no frame)
%\quote{Not Attention, Patience is all we need.}

%----------------------------------------------------------------------------------------

\newcommand{\cvdoublecolumn}[2]{%
  \cvitem[.75em]{}{%
    \begin{minipage}[t]{\listdoubleitemcolumnwidth}#1\end{minipage}%
    \hfill%
    \begin{minipage}[t]{\listdoubleitemcolumnwidth}#2\end{minipage}%
    }%
}



\usepackage{multibbl}
\newcommand\Colorhref[3][orange]{\href{#2}{\small\color{#1}#3}}


% \newcommand{\cvreference}[7]{%
%     \textbf{#1}\newline% Name
%     \ifthenelse{\equal{#2}{}}{}{\addresssymbol~#2\newline}%
%     \ifthenelse{\equal{#3}{}}{}{#3\newline}%
%     \ifthenelse{\equal{#4}{}}{}{#4\newline}%
%     \ifthenelse{\equal{#5}{}}{}{#5\newline}%
%     \ifthenelse{\equal{#6}{}}{}{\emailsymbol~\texttt{#6}\newline}%
%     \ifthenelse{\equal{#7}{}}{}{\phonesymbol~#7}}

\begin{document}

\makecvtitle % Print the CV title




%----------------------------------------------------------------------------------------
%	EDUCATION SECTION
%----------------------------------------------------------------------------------------

\section{Education}

\cventry{2015--2018}{PhD, Electrical Engineering}{Polytechnique Montréal and Université de Bordeaux}{codirec}{Polar Decoding on programmable architectures, \textit{defended on 2018-12-13}}
{Forward Error Correction, Polar Codes, Software Implementations, Hardware Implementations, ASIP}  % Arguments not required can be left empty

\cventry{2012--2015 :}{Master of Engineering, Embedded Electronics}{Bordeaux INP, ENSEIRB-Matmeca}{Bordeaux}{}{}
%{Advanced exposure to various areas of computer science along with a one and half year research project on Reversible Logic Synthesis.}
%\cvitem{CGPA :}{7.96/10}
% \cventry{2009--2013 :}{Bachelor of Engineering, Computer Science \& Technology}{Indian Institute of Engineering Science \& Technology}{Shibpur(\textit{Formerly} Bengal Engineering and Science University, Shibpur)}{}{}
%{Comprehensive exposure to the core areas of Computer Science along with a final year project on Data-mining}
%\cvitem{CGPA :}{7.36/10}
% \cventry{2008 :}{Higher Secondary Examination}{Belmuri Union Institution}{Belmuri}{}{ Mathematics, Physics, Chemistry, Biology, English, Bengali}
% {}
% \cvitem{Percentage :}{81.2 \%}
% \cventry{2006 :}{Secondary Examination}{Belmuri Union Institution}{Belmuri}{}{ Mathematics, Physical Science, Life Science, Geography, History, English, Bengali}
% {}
% \cvitem{Percentage :}{90.8 \%}





%----------------------------------------------------------------------------------------
%	PUBLICATION SECTION
%----------------------------------------------------------------------------------------


\section{Publications}
% \subsection{Journal Article(Accepted)}
% \cventry{2019}{\textbf{Pratik Dutta}, Sriparna Saha, Sanket Pai and Aviral Kumar}{}{Protein-protein Interaction based Generative Model for Improving Gene Clustering}{In \textit{\textbf{Scientific Reports-Nature}} (\textbf{Impact Factor: 4.12)}}{}

\subsection{Journal Articles}
\newbibliography{journal}
\bibliographystyle{journal}{plainyrrev}
\nocite{journal}{*}
\bibliography{journal}{journal}
{\large \textsc{Refereed Journal Articles}}
% \subsection{Communicated Journal Article}
% \cventry{2020}{Pratik Dutta, Aditya Prakash Patra, and Sriparna Saha}{}{DeePROG: An Attention based Deep Multi-modal Architecture for Disease Gene Prognosis}{In \textit{IEEE Transactions on Biomedical Engineering}}{}


\subsection{In Conference Proceedings}
\newbibliography{conference}
\nocite{conference}{*}
\bibliographystyle{conference}{plainyrrev}
\bibliography{conference}{conference}
{\large \textsc{Refereed Conference Publications}}














%----------------------------------------------------------------------------------------
%	WORK EXPERIENCE SECTION
%----------------------------------------------------------------------------------------

\section{Work Experience}
\subsection{ENSEIRB-Matmeca, France}
\cventry{Sep,2018 -- Dec, 2019}{\textit{A Flexible and Portable Real-time DVB-S2 Transceiver using Multicore and SIMD CPUs}}{}{}{}
{Developing a full Software Defined Radio communication chain for real-time processing for satellite communications with Airbus Defense \& Space. 
}
\cvitem{Advisor :}{\textbf{Pr. Christophe Jégo}, \textit{Full Professor, Electrical Engineering Department}, Bordeaux INP ({\Colorhref{https://www.linkedin.com/in/christophe-jego-8b70454/} {\textit{LinkedIn}}})}

% \cventry{July,2018 -- present}{\textit{Developing Deep Multi-modal Architecture for Biomedical Problems}}{}{}{}
% {Analyzing different modalities of genes like gene expression profiles, protein 3D structure, underlying amino acid sequence using popular deep learning models to obtain deeper insight into the underlying biological system. 
% }
% \cvitem{Advisor :}{\textbf{Dr. abc xyz}, \textit{Associate Professor, Department of Computer Science \& Engineering}, IIT abc ({\Colorhref{https://www.personal_webpage.com/} {\textit{Personal Web-page}}})}




\subsection{Worldcast Systems, France}
\cventry{Sep,2012 -- Aug,2015}{\textit{Design and Test of FM transmitters}}{}{}{}{Participated in the design of Ecreso FM transmitters, created a Human-Machine Interface for production and customers.}
\cvitem{Advisor :}{\textbf{Hervé Garat}, \textit{R\&D Engineer},({\Colorhref{https://www.linkedin.com/in/hervegarat/} {\textit{LinkedIn}}})}


% \cventry{September,2012 -- Aug,2015}{\textit{Design and Test of FM transmitters}}{}{}{}{Participated in the design of Ecreso FM transmitters, created a Human-Machine interface for production and customers}
% \cvitem{Advisor :}{\textbf{Hervé Garat}, \textit{Engineer, R&D Departement}, ({\Colorhref{https://www.linkedin.com/in/hervegarat/} {\textit{LinkedIn}}})}












%----------------------------------------------------------------------------------------
%	Fellowships \& Awards
%----------------------------------------------------------------------------------------

% \section{Fellowships \& Awards}

% \cvitem{2016 --present}{\textit{\textbf{Visvesvaraya Fellowship}} of Ministry of Electronics and Information Technology (MeitY), Government of India, as a PhD research scholar in Indian Institute of Technology Patna.}
% \cvitem{2019}{Receipt of \textit{\textbf{Visvesvaraya Travel Grant}} to attend a international conference \textbf{\textit{IEEE Congress on Evolutionary Computation, 2019}} in Wellington, New Zealand.}
% \cvitem{2018}{Recipient of \textit{\textbf{SciGenome Research Foundation (SGRF) GYAN Scholarship}} to participate \textbf{\textit{Nextgen Genomics, Biology, Bioinformatics and Technologies-2018}} meeting at Jaipur India from $30^{th}$ September to $2^{nd}$ October 2018. }
% \cvitem{2015}{Awarded under \textit{\textbf{Students Reward Programme}} at the Annual General Meeting of \textbf{Global Alumni Association of Bengal Engineering and Science University(GAABESU).}}


%----------------------------------------------------------------------------------------
%	Community Involvement
%----------------------------------------------------------------------------------------

\section{Reviewer}


\cvitem{}{IEEE SIPS, IEEE ISTC, GRETSI, IEEE NEWCAS, IEEE SysInt, MDPI Entropy}


%----------------------------------------------------------------------------------------
%	COMPUTER SKILLS SECTION
%----------------------------------------------------------------------------------------

\section{Computer skills}

\cvitem{Programming Languages}{C, C++, Python, PyTorch}
\cvitem{HDL}{VHDL, Vivado HLS}
\cvitem{Software}{Git, Gitlab CI, Linux, Inkscape}


%----------------------------------------------------------------------------------------
%	Position of Responsibility SECTION
%----------------------------------------------------------------------------------------

% \section{Position of Responsibility}
% \cventry{2016-2020}{Executive member of IEEE Student Branch}{}{IIT ABC}{}{}
% \cventry{April 1-5, 2019}{Organizer, GIAN Workshop on subjects}{}{IIT ABC}{}{}

%----------------------------------------------------------------------------------------
%	Research Supervising
%----------------------------------------------------------------------------------------

\section{Research Supervising}
\subsection{Ph.D. students}
\cventry{2020-2023}{Hugo Tessier}{}{IMT Atlantique}{Stellantis}{}
\cventry{2021-present}{Hugo Le Blevec}{}{IMT Atlantique}{}{}
\cventry{2021-present}{Lucas Grativol}{}{IMT Atlantique}{}{}
\subsection{Post-doctoral researchers}
\cventry{2022-present}{Hamoud Younes}{}{IMT Atlantique}{GoodFloow}{}



%----------------------------------------------------------------------------------------
%	Teaching Assistantship SECTION
%----------------------------------------------------------------------------------------

\section{Teaching}
\cventry{2018-2019}{EN112: Digital Electronics Design}{}{ENSEIRB-Matmeca}{}{}
\cventry{2018-2019}{EN102: Combinatorial and Sequential Logic}{}{ENSEIRB-Matmeca}{}{}
\cventry{2018-2019}{EN103: Micro-controller project}{}{ENSEIRB-Matmeca}{}{}
\cventry{2018-2019}{EN114: Computer Architecture}{}{ENSEIRB-Matmeca}{}{}
\cventry{2018-2019}{MI202: Micro-controller project}{}{ENSEIRB-Matmeca}{}{}
\cventry{2018-2019}{PG208: Object-Oriented Programmation with C++}{}{ENSEIRB-Matmeca}{}{}
\cventry{2020-present}{EFFDL: Efficient Deep Learning}{}{IMT Atlantique}{}{}
\cventry{2020-present}{SEIML: Embedded Systems - Software Hardware Interaction}{}{IMT Atlantique}{}{}
\cventry{2020-present}{ParPIng: Parallel Computing for Engineers}{}{IMT Atlantique}{}{}

% %----------------------------------------------------------------------------------------
% %	Fundings Obtained
% %----------------------------------------------------------------------------------------

\section{Fundings Obtained}

\cventry{2023-2026}{ANR JCJC}{250k€}{ProPruNN: Profitable Pruning of Neural Networks}{Project Lead}{ANR}
\cventry{2022-2024}{Labex CominLabs}{325k€}{Leasard: Low Energy deep neural networks for Autonomous Search-And-Rescue Drones}{Member}{}
\cventry{2022-2024}{AI@IMT}{120k€}{Leasard: Low Energy deep neural networks for Autonomous Search-And-Rescue Drones}{Member}{IMT}
\cventry{2022-2024}{GDR ISIS}{7k€}{Furnitures}{}{CNRS}
\cventry{2022}{Maupertuis visit program}{1k€}{}{}{Institut Français Finland}
\cventry{2021-2024}{Futur et Ruptures}{120k€}{FLCNNFPGA: Towards an efficient and privacy-protecting IoT through the use of federated learning and FPGA technologies}{Member}{IMT Atlantique}

% %----------------------------------------------------------------------------------------
% %	Visiting Researcher
% %----------------------------------------------------------------------------------------

\section{Visiting Researcher}
\subsection{As Guest}
\cventry{2022}{Tampere University}{Finland}{}{}{}

% \cvitem{2022}{\textbf{Project Leader of ANR JCJC Grant ``ProPruNN: Profitable Pruning Of Neural Network''}.}


% \cvitem{2018}{Invited to conduct lab sessions in \textit{\textbf{"Training Program on Machine Learning For Ocean Acoustics and Climate Data Analysis"}}, during 22-36 October 2018 at \textbf{Defence R\&D Organization- Naval Physical \& Oceanographic Laboratory (DRDO-NPOL), Kochi, Kerala}.}



% \section{Referees}


% \begin{tabular}{lr}
% % Referee 1
% \begin{minipage}[t]{3in}
% \textbf{Pr. Christophe Jégo}\\
% \textit{Full Professor , Department of} \\
% \textit{Computer Science \& Engineering}\\
% Institute name\\
% \Letter\ \href{mailto:abc@gmail.com}{abc@gmail.com}
% \end{minipage}
% &
% % Referee 2
% \begin{minipage}[t]{3in}
% \textbf{Dr. XXXXX XXXXX}\\
% \textit{Associate Professor, Department of} \\
% \textit{Computer Science \& Engineering}\\
% Institute name\\
% \Telefon\ +(601) 877-6236\\
% \Letter\ \href{mailto:abc@gmail.com}{abc@gmail.com}
% \end{minipage}
% \\
% \\ % Additional newline for spacing.
% % Referee 3
% \begin{minipage}[t]{3in}
% \textbf{Dr. XXXXX XXXXX}\\
% \textit{Associate Professor, Department of} \\
% \textit{Computer Science \& Engineering}\\
% Institute name\\ 
% \Telefon\ +(601) 877-6236\\
% \Letter\ \href{mailto:abc@gmail.com}{abc@gmail.com}
% \end{minipage}
% &
% \\
% \end{tabular}


\end{document}
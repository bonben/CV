%%%%%%%%%%%%%%%%%%%%%%%%%%%%%%%%%%%%%%%%%
% "ModernCV" CV and Cover Letter
% LaTeX Template
% Version 1.1 (9/12/12)
%
% This template has been downloaded from:
% http://www.LaTeXTemplates.com
%
% Original author:
% Xavier Danaux (xdanaux@gmail.com)
%
% License:
% CC BY-NC-SA 3.0 (http://creativecommons.org/licenses/by-nc-sa/3.0/)
%
% Important note:
% This template requires the moderncv.cls and .sty files to be in the same
% directory as this .tex file. These files provide the resume style and themes
% used for structuring the document.
%
%%%%%%%%%%%%%%%%%%%%%%%%%%%%%%%%%%%%%%%%%

%----------------------------------------------------------------------------------------
%	PACKAGES AND OTHER DOCUMENT CONFIGURATIONS
%----------------------------------------------------------------------------------------
\documentclass[11pt,a4paper,sans]{moderncv}
\usepackage{standalone}
\moderncvstyle{classic}
\moderncvcolor{blue}
\usepackage{lipsum}
\usepackage[scale=0.85]{geometry}
\usepackage{fontawesome}
\usepackage{marvosym}

%----------------------------------------------------------------------------------------
%	SECTION NOM ET INFORMATIONS DE CONTACT
%----------------------------------------------------------------------------------------

\firstname{MATHIEU}
\familyname{LEONARDON}

\title{Maître de conférences\\ \\ Adéquation Algorithme-Architecture \\ Compression en Deep Learning}
\address{Département MEE}{IMT Atlantique\\ né le 18 mars 1987, français\\}
\mobile{(+33) 229001384}
\email{mathieu.leonardon@imt-atlantique.fr}
\homepage{www.mathieuleonardon.com/}{Ma page web}

\extrainfo{
    \faGithub\href{https://github.com/bonben}{ Github} \quad
    \faLinkedin\href{https://bit.ly/3us0P9P}{ Linkedin} \quad
    }
\photo[70pt][0.3pt]{picture}

\newcommand{\cvdoublecolumn}[2]{%
  \cvitem[.75em]{}{%
    \begin{minipage}[t]{\listdoubleitemcolumnwidth}#1\end{minipage}%
    \hfill%
    \begin{minipage}[t]{\listdoubleitemcolumnwidth}#2\end{minipage}%
    }%
}

\usepackage{multibbl}
\newcommand\Colorhref[3][orange]{\href{#2}{\small\color{#1}#3}}

\begin{document}

\makecvtitle

%----------------------------------------------------------------------------------------
%	SECTION FORMATION
%----------------------------------------------------------------------------------------

\section{Formation}

\cventry{2015--2018}{Doctorat, Génie électrique}{Polytechnique Montréal et Université de Bordeaux}{codirec}{Décodage de codes polaires sur des architectures programmables, \textit{soutenue le 2018-12-13}}
{}
\cvitem{Président}{Mohamad Sawan}
\cvitem{Rapporteurs}{Amer Baghdadi, Emmanuel Casseau}
\cvitem{Directeurs}{Christophe Jégo, Yvon Savaria}
\cvitem{Examinateurs}{Camille Leroux, Olivier Muller, Charly Poulliat}
\cventry{2012--2015}{Master en Ingénierie, Électronique Embarquée}{Bordeaux INP, ENSEIRB-Matmeca}{Bordeaux}{}{}

%----------------------------------------------------------------------------------------
%	SECTION PUBLICATIONS
%----------------------------------------------------------------------------------------

\section{Publications}

\subsection{Articles de revue}
\newbibliography{journal}
\bibliographystyle{journal}{plainyrrev}
\nocite{journal}{*}
\bibliography{journal}{journal}
{\large \textsc{Articles de revue avec comité de lecture}}

\subsection{Actes de conférences}
\newbibliography{conference}
\nocite{conference}{*}
\bibliographystyle{conference}{plainyrrev}
\bibliography{conference}{conference}
{\large \textsc{Publications en conférences avec comité de lecture}}

%----------------------------------------------------------------------------------------
%	SECTION EXPÉRIENCE PROFESSIONNELLE
%----------------------------------------------------------------------------------------

\section{Expérience Professionnelle}

\cventry{2020 -- présent}{Maître de conférences}{IMT Atlantique, Lab-STICC, UMR CNRS 6285}{}{}{}
\cventry{2018 -- 2019}{ATER}{Bordeaux INP, ENSEIRB-Matmeca France}{}{}{}
\cventry{2015 -- 2018}{Doctorant}{Bordeaux INP, ENSEIRB-Matmeca France}{et \textit{Polytechnique Montréal - Canada}}{}{}
\cventry{2012 -- 2015}{Ingénieur en apprentissage}{Worldcast Systems, France}{}{}{}

%----------------------------------------------------------------------------------------
%	SECTION IMPLICATION DANS LA COMMUNAUTÉ
%----------------------------------------------------------------------------------------

\section{Évaluateur scientifique}

\cvitem{}{IEEE SIPS, IEEE ISTC, GRETSI, IEEE NEWCAS, IEEE SysInt, MDPI Entropy, Neurips}

\section{Compétences informatiques}

\cvitem{Langages de programmation}{C, C++, Python, PyTorch}
\cvitem{HDL}{VHDL, Vivado HLS}
\cvitem{Logiciels}{Git, Gitlab CI, Linux, Inkscape}

%----------------------------------------------------------------------------------------
%	SECTION RESPONSABILITÉS
%----------------------------------------------------------------------------------------

%----------------------------------------------------------------------------------------
%	Encadrement de recherche
%----------------------------------------------------------------------------------------

\section{Encadrement de recherche}
\subsection{Doctorants}
\cventry{2020-2023}{Hugo Tessier}{}{IMT Atlantique}{Stellantis}{}
\cventry{2021-2024}{Hugo Le Blevec}{}{IMT Atlantique}{}{}
\cventry{2021-2024}{Lucas Grativol}{}{IMT Atlantique}{Télécom Saint-Étienne}{}
\cventry{2022-présent}{Timotée Ly-Manson}{}{IMT Atlantique}{}{}
\cventry{2022-présent}{Karl La Grassa}{}{IMT Atlantique}{Polytechnique Montréal}{}
\cventry{2022-présent}{Ismail Amessegher}{}{IMT Atlantique}{University of Adelaide}{}
\cventry{2025-présent}{Elise Lagarde}{}{IMT Atlantique}{Télécom Paris}{Arkane}
\subsection{Post-doctorants}
\cventry{2022-2023}{Hamoud Younes}{}{IMT Atlantique}{GoodFloow}{}
\cventry{2023-présent}{Hugo Tessier}{}{IMT Atlantique}{}{}
\cventry{2024-présent}{Hugo Le Blevec}{}{IMT Atlantique}{}{}
\subsection{Ingénieur de recherche}
\cventry{2024-2025}{Manon Renault}{}{IMT Atlantique}{}{}

%----------------------------------------------------------------------------------------
%	SECTION ENSEIGNEMENT
%----------------------------------------------------------------------------------------

\section{Enseignement}
\cventry{2018-2019}{EN112 : Conception électronique numérique}{}{ENSEIRB-Matmeca}{}{}
\cventry{2018-2019}{EN102 : Logique combinatoire et séquentielle}{}{ENSEIRB-Matmeca}{}{}
\cventry{2018-2019}{EN103 : Projet microcontrôleur}{}{ENSEIRB-Matmeca}{}{}
\cventry{2018-2019}{EN114 : Architecture des ordinateurs}{}{ENSEIRB-Matmeca}{}{}
\cventry{2018-2019}{MI202 : Projet microcontrôleur}{}{ENSEIRB-Matmeca}{}{}
\cventry{2018-2019}{PG208 : Programmation orientée objet avec C++}{}{ENSEIRB-Matmeca}{}{}
\cventry{2020-présent}{EFFDL : Deep Learning efficace}{}{IMT Atlantique}{Responsable}{}
\cventry{2020-présent}{SEIML : Systèmes embarqués - Interaction logiciel/matériel}{}{IMT Atlantique}{}{}
\cventry{2022-2025}{SECTI : Du capteur au traitement intelligent}{}{IMT Atlantique}{}{}
\cventry{2020-présent}{ParPIng : Calcul parallèle pour ingénieurs}{}{IMT Atlantique}{Responsable}{}
\cventry{2020-2022}{CHLS : Conception de circuits haut niveau}{}{IMT Atlantique}{}{}
\cventry{2020-2022}{Électronique}{}{IMT Atlantique}{}{}
\cventry{2020-2025}{Architecture des ordinateurs}{}{IMT Atlantique}{}{}
\cventry{2020-2025}{Fonctions électroniques logiques et analogiques}{}{IMT Atlantique}{}{}
\cventry{2020-2025}{Divers projets étudiants}{}{IMT Atlantique}{}{}

%----------------------------------------------------------------------------------------
%	SECTION FINANCEMENTS PUBLICS OBTENUS
%----------------------------------------------------------------------------------------

\section{Financements publics obtenus}
\cventry{2024-2026}{PCR}{140k€}{Optimisation du traitement spatio-temporel pour l'antibrouillage GNSS avec antennes CRPA utilisant des beamformers CNN et TCN légers}{Membre}{Région Bretagne}
\cventry{2023-2026}{ANR JCJC}{250k€}{ProPruNN : Élagage rentable de réseaux de neurones}{Chef de projet}{ANR}
\cventry{2023-2026}{ANR TSIA}{200k€}{Vortex : Essaims de drones reconfigurables basés sur la vision pour une exploration rapide}{Membre}{ANR}
\cventry{2024-2025}{R\&T CNES}{5k€}{Deep-Learning-FPGA-Challenge d’éco-conception}{Chef de projet}{CNES}
\cventry{2022-2024}{Labex CominLabs}{325k€}{Leasard : Réseaux neuronaux profonds basse consommation pour drones autonomes de recherche et sauvetage}{Membre}{}
\cventry{2022-2024}{AI@IMT}{120k€}{Leasard : Réseaux neuronaux profonds basse consommation pour drones autonomes de recherche et sauvetage}{Membre}{IMT}
\cventry{2022-2024}{GDR ISIS}{7k€}{Mobilier}{}{CNRS}
\cventry{2022}{Programme de visite Maupertuis}{1k€}{}{}{Institut Français Finlande}
\cventry{2021-2024}{Futur et Ruptures}{120k€}{FLCNNFPGA : Vers un IoT efficace et respectueux de la vie privée grâce à l’apprentissage fédéré et aux FPGA}{Membre}{IMT Atlantique}

%----------------------------------------------------------------------------------------
%	SECTION FINANCEMENTS INDUSTRIELS OBTENUS
%----------------------------------------------------------------------------------------

\section{Financements industriels obtenus}
\cventry{2025-2028}{Thèse CIFRE}{}{Optimisation du traitement spatio-temporel pour l’antibrouillage GNSS des antennes CRPA via des beamformers CNN et TCN légers}{Encadrant}{Arkane}
\cventry{2023-2025}{Contrat de recherche}{}{Analyse du Few-Shot Neural Architecture Search dans un cadre métrique}{Membre}{Sony}
\cventry{2024-2025}{Contrat de recherche}{}{Deep Learning embarqué pour la segmentation sémantique et les LLM}{Chef de projet}{Schneider Electric}
\cventry{2020-2023}{Thèse CIFRE}{}{Élagage de réseaux de neurones convolutionnels et application aux systèmes de vision embarqués}{Encadrant}{Stellantis}
\cventry{2022}{Contrat de recherche}{}{DeepGEMM : Inférence ultra basse précision accélérée sur architectures CPU utilisant des tables de consultation}{Chef de projet}{Deeplite}

%----------------------------------------------------------------------------------------
%	SECTION CHERCHEUR INVITÉ
%----------------------------------------------------------------------------------------

\section{Chercheur invité}
\subsection{En tant qu’invité}
\cventry{2023}{Université Tohoku}{Japon}{1 mois}{}{}
\cventry{2022}{Université de Tampere}{Finlande}{2 semaines}{}{}
\cventry{2022}{Sony Stuttgart}{Allemagne}{2 semaines}{}{}



\end{document}